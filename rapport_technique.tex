\documentclass{rapport}

\graphicspath{{Images/}}

\begin{document}
\begin{titlepage}
\begin{center}

% Upper part of the page. The '~' is needed because only works if a paragraph has started.
\includegraphics[width=0.35\textwidth]{logo.png}~\\[1cm]

\textsc{\LARGE Université de Franche-Comté}\\[1.5cm]

\textsc{\Large }\\[5cm]

% Title
\textbf{\HRule }\\[0.4cm]

{\huge \bfseries Rapport Technique\\
Photoelectric effect on interstellar grain\\[0.4cm] }

\HRule \\[3cm]
% Author and supervisor
\begin{minipage}{0.4\textwidth}
\begin{flushleft} \large
\emph{Auteur:}\\
\textsc{Nicolas BELLEMONT}\\
\end{flushleft}
\end{minipage}
\begin{minipage}{0.4\textwidth}
\begin{flushright} \large
\emph{Date de fin de développement:} \\
\textsc{07/10/2018 Version : Gold V.2}\\
\emph{Language de programmation:} \\
\textsc{Python}
\end{flushright}
\end{minipage}


\vfill

% Bottom of the page
{\large \today}

\end{center}
\end{titlepage}
\newpage

\section{Liste des fichiers}
\begin{itemize}
  \item Simulation.py : fichier principal, à exécuter pour lancer les simulations.
  \item Functions.py : fichier contenant toutes les fonctions utilisés par le programme.
  \item Constant.py : fichier contenant les valeurs des constantes necessaire au programme.
\end{itemize}






\section{Spécification fonctionnelle du programme}

\subsection{But du programme}
Le programme présenté dans ce rapport à pour but d'étudier l'effet photoélectrique sur un grain de matière interstellaire.\nl
Son fonctionnement se découpe en deux étapes, d'abord il génère une matrice de 0 et de 1 représentant le grain de poussière(que l'ont assimilera à du carbone), puis il simule l'arrivé sur ce grain d'un certains nombres de photon interstellaire. \nl
Au final ce programme permet de voir l'influence de la géométrie d'un grain de poussière sur l’effet photoélectrique.

\subsection{Données d'entrées}
\begin{itemize}
  \item number\_of\_grain : entier, détermine combien de grain seront générés pour les différentes exploitation.
  \item sigma : réel, représente la densité de la distribution du grain (en pratique sers à jouer sur le degré de "fractalité" d'un grain).
  \item N : entier, nombre de photon à envoyer sur les grain lors des simulations.
\end{itemize}

\subsection{Sortie du programme}
Le programme ressort une série d'histogramme correspondant à la simulation choisi par l'utilisateur, ces histogramme représentent le nombre de photo-électron qui se sont échappés du grain avec les énergies correspondantes.

\subsection{Conditions spécifique d'utilisation}



\section{Fonctionnement interne du programme}

\subsection{Description des modèles physiques}


\subsection{Description des algoritmes de calculs scientifiques utilisés}


\subsection{Liste des éléments costitutif du programme}


\subsection{Diagramme fonctionnel du programme}





\section{Fiabilité du programme, démarche qualité}
\section{Exemples de résultats du programme}





















\end{document}
